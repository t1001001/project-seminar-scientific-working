\section{Limitations and Conclusion}
This section summarizes the key limitations of our study and hightlights their implications for generalizability and interpretability. We describe the dataset, models, and constraints and discuss how these factors may influence our results. Finally, we outline directions for potential future work to mitigate these limitations.
\subsection{Limitations}
This work has several limitations. The evaluation is limited to a single dataset (LUNA16), a single GAN architecture (CycleGAN) and only one object detection model (YOLO11), which limits the generalizability of the findings. \\
Furthermore, the experiments are conducted on two-dimensional CT slices rather than full three-dimensional slices, which may limit the ability of both classical and GAN-based augmentation methods to capture inter-slice contextual information relevant for pulmonary nodule detection. \\
Additionaly, while CycleGAN-based augmentation increases appearance variability, it does not introduce geometric variations, which are critical for accurate object localization and may reduce their usefulness for improving such localization. \\
Beyond that, the realism of GAN-generated images is not assessed by clinical experts, although just appearance-level shifts, leaves the perceptual and diagnostic validity of the synthetic images unverified. \\
Beyond these limitations, computational contraints such as training CycleGAN on a capped number of images and only using subsets of the original dataset may have underutilized the potential of GAN-based augmentation. \\
Finally, while fixed hyperparameters and a single validation split, while ensuring comparability, may not reflect optimal hyperparameters or dataset variability. \\
\subsection{Conclusion}
This study evaluated CycleGAN-based augmentation for pulmonary nodule detection in chest CT scans using YOLO11 across four controlled training settings. The inclusion of CycleGAN-generated images increased AP@0.5:0.95 and Mean IoU relative to the baseline, but with a marginal decrease in AP@0.5. Appearance-level, label-preserving augmentations improved all metrics over baseline, and their combination with CycleGAN achieved the best, yet modest results. \\
These findings indicate that appearance-level synthesis via CycleGAN offers little benefit for object detection and that gains observed in classification and segmentation does not necessarily translate into gains in object detection, and is difficult to justify given its computational overhead when its effect overlap with classical augmentations. \\
Future work should explore anatomically consistent, diffusion- or 3D style-based generators to better capture geometric variability for precise detection. \\
Limitations include reliance on a single dataset, one GAN architecture, and one object detection model, 2D slice processing, the absence of clinical validation of realism and computational constraints.