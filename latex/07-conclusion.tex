\section{Conclusion}
This study evaluated CycleGAN-based augmentation for pulmonary nodule detection in chest CT scans using YOLO11 across four controlled training settings. The inclusion of CycleGAN-generated images increased AP€0.5:0.95 and Mean IoU relative to the baseline, but with a slight decrease in AP@0.5. Appearance-level, label-preserving augmentations improved all metrics over baseline, and their combination with CycleGAN achieved the best, yet modest results. \\
These findings indicate that appearance-level synthesis via CycleGAN offers little benefit for object detection and that gains observed in classification and segmentation does not necessarily translate into gains in object detection. \\
Future work should explore anatomically consistent, diffusion- or 3D style-based generators to better capture geometric variability for precise detection. \\
Limitations include reliance on a single dataset, one GAN architecture, and one object detection model, as well as 2D slice processing and absence of clinical validation of realism.