% This is samplepaper.tex, a sample chapter demonstrating the
% LLNCS macro package for Springer Computer Science proceedings;
% Version 2.21 of 2022/01/12
%
\documentclass[runningheads]{llncs}
%
\usepackage[T1]{fontenc}
% T1 fonts will be used to generate the final print and online PDFs,
% so please use T1 fonts in your manuscript whenever possible.
% Other font encondings may result in incorrect characters.
%
\usepackage{graphicx}
\usepackage{float}
% Used for displaying a sample figure. If possible, figure files should
% be included in EPS format.
%
% If you use the hyperref package, please uncomment the following two lines
% to display URLs in blue roman font according to Springer's eBook style:
%\usepackage{color}
%\renewcommand\UrlFont{\color{blue}\rmfamily}
%\urlstyle{rm}
%
\begin{document}
%
\title{Enhancing Medical Object Detection through GAN-generated CT Images}
%
%\titlerunning{Abbreviated paper title}
% If the paper title is too long for the running head, you can set
% an abbreviated paper title here
%
\author{Tobias Nguyen\inst{1} \and
Cumali Karaali\inst{2}}
%
\authorrunning{Nguyen \& Karaali}
% First names are abbreviated in the running head.
% If there are more than two authors, 'et al.' is used.
%
\institute{Aalen University, Beethovenstr. 1, 73430 Aalen \\
\email{88299@studmail.htw-aalen.de} \and
Aalen University, Beethovenstr. 1, 73430 Aalen \\
\email{88154@studmail.htw-aalen.de}}
%
\maketitle              % typeset the header of the contribution
%
\begin{abstract}
The limited availability of annotated medical imaging data is a major challenge for deep-learning based object detection. Generative Adversarial Networks have been proposed as a potential solution by generating synthetic images for data augmentation and have shown promising results in medical image classification and segmentation. However, their effectiveness in object detection remains underexplored.

This work investigates the impact of CycleGAN-based data augmentation on pulmonary nodule detection in chest CT images. Experiments are conducted on the LUNA16 dataset using the YOLO11 object detection framework. YOLO11 is trained on real data only is compared to YOLO11 trained on a combination of real and GAN-generated images and purely GAN-generated images. Detection performance is evaluated using Average Precision and Intersection over Union.

Preliminary results show that the inclusion of CycleGAN-generated images does not improve detection performance and leads to slightly reduced localization accuracy. These findings suggest that augmentation strategies effective for classification and segmentation may not transfer to object detection.
\keywords{Medical Object Detection, Generative Adversarial Networks, CycleGAN, YOLO, Data Augmentation}
\end{abstract}
%
%
%
% ------------------------------------------------------------
\section{Introduction} % Cumi
Object detection in computed tomography (CT) plays a crucial role in reinforcing diagnostic accuracy and clinical decision-making. In the last few years, deep learning-based object detection frameworks like \textbf{You Only Look Once (YOLO)} have shown solid performance in many imaging tasks because of their efficiency and real-time detection capabilities \cite{yolo11}\cite{yoloreview}. However, object detection models, especially in sensitive fields like medical imaging, often face the problem of having too little usable labeled data, or it is very difficult to access such data due to privacy regulations, costs, and other restrictions. As a result, models trained on small or unbalanced datasets tend to overfit and perform poorly when generalizing to new test data. To address this issue, we make use of \textbf{Generative Adversarial Networks (GANs)}, introduced by Goodfellow et al. \cite{goodfellow2014gan}, which can generate synthetic data to improve model training. Researchers such as Frid-Adar and Sandfort have shown that GAN-based augmentation can significantly improve classification and segmentation performance in CT-imaging\cite{frid2018gan}\cite{sandfort2019data}. CycleGAN-based augmentation has also been applied to YOLO-based multi-organ detection in CT scans \cite{cycleganeaugmentation}. 
\subsection{Novelty \& Expected Contributions} % Cumi
This paper aims to support the medical sector in data analysis by analyzing how GAN-based data augmentation affects object detection in CT imaging. The novelty of this work lies in a comparative approach, analyzing the performance differences between object detection models trained on original datasets and those trained on datasets augmented with GAN-generated CT images. The expected contributions of this work include:
\begin{itemize}
\item[--] Empirical evaluation of GAN-generated CT images for improving object detection model performance.

\item[--] Quantitative comparison between models trained on original, augmented and purely synthetic datasets using standard object detection metrics.

\item[--] Insights into the generalization potential of GAN-based augmentation for limited medical datasets.

\item[--] Guidelines for future research on integrating synthetic data into medical AI pipelines.
\end{itemize}
\subsection{Research Question} % Tobi
This work addresses the following question: \\
\textbf{Do CycleGAN-generated chest CT images improve lung nodule detection accuracy when real training data is limited?}

% ------------------------------------------------------------
\section{Related Work}

\subsection{GAN-based Augmentation in Medical Imaging}
Generative Adversarial Networks (GANs)~\cite{goodfellow2014gan} and CycleGAN~\cite{zhu2017cyclegan} are widely used for medical image synthesis and augmentation. Empirical evidence shows substantial gains in CT-based classification and segmentation when augmenting with GAN-generated images~\cite{frid2018gan,sandfort2019cycleganCT}. However, transferring these benefits to spatially sensitive object detection is less clear. CycleGAN-based augmentation for YOLO in multi-organ CT detection reported mixed outcomes~\cite{hammami2020cyclegan}, suggesting that appearance-level transformations may not consistently enhance bounding-box localization. \\
Beyond GANs, diffusion models and style-based generators provide additional context. Transformer- and latent-diffusion approaches improve anatomical consistency and downstream performance in multiple modalities~\cite{pan2023ddpm,khader2023ddpm3d}. 3D and StyleGAN-based methods have modeled lung CT textures with high realism~\cite{ellis2022stylegan3d,shi2020stylegan}. These works indicate that volumetric or anatomy-aware synthesis may be more suitable than purely appearance-level augmentation for tasks requiring precise localization.

\subsection{Object Detection in Chest CT}
Detectors such as Faster R-CNN~\cite{ren2015fasterrcnn}, RetinaNet~\cite{lin2017focal}, and the YOLO family~\cite{redmon2016yolo} are widely adopted in medical object detection. LUNA16 remains a standard benchmark for pulmonary nodule detection in CT~\cite{setio2017luna16}. Evaluation commonly relies on AP/mAP and IoU~\cite{padilla2020metrics}. Systematic reviews highlight extensive use of YOLO in medical imaging and continued architectural advances~\cite{ragab2024yoloreview}. Recent CT-specific works demonstrate strong baselines via multi-scale receptive fields, transfer learning, and data-centric tuning~\cite{wu2024yolomsrf,harsono2022retinanet,nguyen2022datacentric,wehbe2024yolov8}.

\subsection{Positioning of This Work}
Prior literature establishes the utility of GAN augmentation primarily for classification and segmentation~\cite{frid2018gan,sandfort2019cycleganCT}, while evidence for CT-based object detection remains limited and inconclusive~\cite{hammami2020cyclegan}. This paper addresses that gap through a controlled comparison of four training settings (baseline, classical augmentation, baseline + CycleGAN, classical augmentation + CycleGAN) under identical hyperparameters and evaluation metrics (AP@0.5, AP@0.5:0.95, Mean IoU). Our findings provide insights into the limitations of CycleGAN-based augmentation for medical object detection.

% ------------------------------------------------------------
\section{Methodology} % Tobi
\subsection{Dataset description} % Tobi
The LUNA16 dataset is a benchmark medical imaging dataset for lung nodule detection. It consists of 888 chest CT scans and is derived from the LIDC-IDRI dataset~\cite{luna16}.
\subsection{GAN Model} % Tobi
We utilize a CycleGAN architecture to generate realistic synthetic chest CT images for data augmentation. CycleGAN consists of two generator and discriminator networks, enabling unpaired image-to-image translation~\cite{zhu2020unpairedimagetoimagetranslationusing}. CycleGAN has shown strong performance in medical image synthesis tasks~\cite{cycleganeaugmentation}\cite{sandfort2019data}.
\subsection{Object Detection Model} % Tobi
For the object detection task, we utilize the YOLO11 architecture, which represents the latest advancement in the YOLO familiy~\cite{yolo11}. YOLO has demonstrated strong capabilities in various medical imaging tasks, making it suitable for this work~\cite{yoloreview}.
\subsection{Training Procedure} % Tobi
The training process consists of two stages: (1) GAN training and (2) object detector training. First, the CycleGAN is trained on the LUNA16 dataset to generate synthetic chest CT images. The synthetic images are combined with the original images to create a augmented dataset. Next, YOLO11 is trained in three settings: (1) using only the original LUNA16 dataset, (2) using the augmented dataset consisting of original and GAN-generated images and (3) only using GAN-generated CT images. To increase generalizability, additional data augmentation such as rotation, scaling, and noise injection are applied in the training process.
\subsection{Evaluation} % Tobi
The performance of YOLO11 is evaluated using standard metrics for object detection~\cite{odmetrics}. These include:
\begin{itemize}
    \item \textbf{Intersection over Union:} Measures the overlap between two regions by dividing the area of their intersection by the area of their union.
    \item \textbf{Average Precision:} Measures the accuracy by calculating the area under the precision-recall curve.
\end{itemize}
% ------------------------------------------------------------
\section{Results}
This section summarizes the qualitative examples and quantitative performance of the four training settings and show the effects of CycleGAN and classical augmentation on object detection.

\subsection{Qualitative Examples}
Figure~\ref{fig:aug_examples} shows appearance-level augmentations (contrast, brightness, blur, noise, CLAHE) applied to a representative slice; Figure~\ref{fig:cyclegan_example} shows a CycleGAN-generated sample.
\begin{figure}[H]
\centering
\begin{subfigure}{0.3\linewidth}
  \includegraphics[width=\linewidth]{figs/1.3.6.1.4.1.14519.5.2.1.6279.6001.100621383016233746780170740405_0142.png}
  \caption{Original}
\end{subfigure}\hfill
\begin{subfigure}{0.3\linewidth}
  \includegraphics[width=\linewidth]{figs/1.3.6.1.4.1.14519.5.2.1.6279.6001.100621383016233746780170740405_0142_aug1.png}
  \caption{Contrast}
\end{subfigure}\hfill
\begin{subfigure}{0.3\linewidth}
  \includegraphics[width=\linewidth]{figs/1.3.6.1.4.1.14519.5.2.1.6279.6001.100621383016233746780170740405_0142_aug2.png}
  \caption{Brightness}
\end{subfigure}
\vspace{0.6em}
\begin{subfigure}{0.3\linewidth}
  \includegraphics[width=\linewidth]{figs/1.3.6.1.4.1.14519.5.2.1.6279.6001.100621383016233746780170740405_0142_aug3.png}
  \caption{Gaussian blur}
\end{subfigure}\hfill
\begin{subfigure}{0.3\linewidth}
  \includegraphics[width=\linewidth]{figs/1.3.6.1.4.1.14519.5.2.1.6279.6001.100621383016233746780170740405_0142_aug4.png}
  \caption{Gaussian noise}
\end{subfigure}\hfill
\begin{subfigure}{0.3\linewidth}
  \includegraphics[width=\linewidth]{figs/1.3.6.1.4.1.14519.5.2.1.6279.6001.100621383016233746780170740405_0142_aug5.png}
  \caption{CLAHE}
\end{subfigure}
\caption{Original CT slices including appearance-level augmentations.}
\label{fig:aug_examples}
\end{figure}
\begin{figure}[H]
\centering
\begin{subfigure}{0.3\linewidth}
    \includegraphics[width=\linewidth]{figs/cyc_1.3.6.1.4.1.14519.5.2.1.6279.6001.100621383016233746780170740405_0142_fake_A.png}
    \caption{Fake A}
\end{subfigure}\hspace{0.05\linewidth}
\begin{subfigure}{0.3\linewidth}
    \includegraphics[width=\linewidth]{figs/cyc_1.3.6.1.4.1.14519.5.2.1.6279.6001.100621383016233746780170740405_0142_fake_B.png}
    \caption{Fake B}
\end{subfigure}
\caption{CycleGAN-generated CT slices.}
\label{fig:cyclegan_example}
\end{figure}

\subsection{Quantitative Performance}
Four settings are evaluated on the fixed validation split using AP@0.5, AP@0.5:0.95, and Mean IoU. \\
\begin{table}[htbp]
\centering
\caption{Validation performance and data composition per training setting. \\
 $N_{\text{real}}$ = number of real training images (unique). \\
 $N_{\text{aug}}$ = number of classical augmentation copies. \\
 $N_{\text{gan}}$ = number of CycleGAN images.}
\begin{tabular}{lrrrrrr}
\toprule
Setting & $N_{\text{real}}$ & $N_{\text{aug}}$ & $N_{\text{gan}}$ & AP@0.5 & AP@0.5:0.95 & Mean IoU \\
\midrule
Baseline & 295 & 0 & 0 & 0.711 & 0.384 & 0.695 \\
Baseline + Classical Aug & 295 & $5\times$295 & 0 & 0.753 & 0.431 & 0.761 \\
Baseline + CycleGAN & 295 & 0 & 100 & 0.724 & 0.383 & 0.708 \\
Classical Aug + CycleGAN & 295 & $5\times$295 & 100 & \textbf{0.822} & \textbf{0.476} & \textbf{0.763} \\
\bottomrule
\end{tabular}
\end{table}
Across all four training settings, classical augmentations consistently improved all metrics relative to the baseline trained solely on real slices. In particular, adding classical augmentations led to higher AP@0.5, AP@0.5:0.95, and Mean IoU. \\
Training with CycleGAN-generated images (in addition to real images) yielded a higher AP@0.5:0.95 and Mean IoU compared to baseline, while AP@0.5:0.95 decreased (0.383 vs. 0.384), although this difference is so small that is likely falls withing normal variation. \\
The combined setting of classical augmentation + CycleGAN achieved the best overall performance across all metrics, though the gains over classical augmentation alone were modest. \\
Altogether, these results suggest that CycleGAN serves as a complementary source of appearance variability that can modestly improve performance but does not replace classical augmentation. \\
Given that classical augmentation also improved all metrics at lower computational overhead, the additional computational complexity of CycleGAN is hard to justify if solely used for appearance robustness.
% ------------------------------------------------------------
\section{Discussion}
Across all training settings, CycleGAN-based augmentation provides limited benefits for pulmonary nodule detection. Compared to the baseline setting, CycleGAN increased AP@0.5:0.95 and Mean IoU, with a small decrease in AP@0.5. Classical augmentations also improved all metrics over baseline, and the combination of classical augmentation + CycleGAN achieved the strongest results, albeit modestly. \\
The results suggest that CycleGAN acts as a complementary source of appearance variability rather than a replacement of classical augmentation. Compared to the baseline setting it increases AP@0.5:0.95 and Mean IoU, yet the largest gain comes when CycleGAN is combined with classical augmentations, which remain the largest source of improvements. \\
These findings align with prior reports that appearance-level synthesis may not fully capture the geometric variability critical for precise localization~\cite{hammami2020cyclegan}, limiting gains for object detection. In line with this, anatomically consistent synthesis, such as diffusion or 3D style-based generators could be promising directions~\cite{khader2023ddpm3d,ellis2022stylegan3d}. \\
Altogether, these results suggest that CycleGAN-based augmentation provides limited benefit for object detection in chest CT scans and that GAN-based augmentation strategies effective for classification and segmentation tasks may not transfer to object detection tasks.
% ------------------------------------------------------------
\section{Limitations}
This work has several limitations. The evaluation is limited to a single dataset (LUNA16), a single GAN architecture (CycleGAN) and only one object detection model (YOLO11), which limits the generalizability of the findings. \\
Furthermore, the experiments are conducted on two-dimensional CT slices rather than full three-dimensional slices, which may limit the ability of both classical and GAN-based augmentation methods to capture inter-slice contextual information relevant for pulmonary nodule detection. Additionaly, while CycleGAN-based augmentation increases appearance variability, it does not explicitly model geometric variations, which are critical for accurate object localization and may reduce their usefulness for improving such localization. Furthermore, no statistical significance testing was conducted, and the observed performance differences may fall within the variance introduced by random initialization. \\
Lastly, the realism of GAN-generated images is not assessed by clinical experts.
% ------------------------------------------------------------
\section{Conclusion}
This study evaluated CycleGAN-based augmentation for pulmonary nodule detection in chest CT scans using YOLO11 across four controlled training settings. The inclusion of CycleGAN-generated images increased AP€0.5:0.95 and Mean IoU relative to the baseline, but with a slight decrease in AP@0.5. Appearance-level, label-preserving augmentations improved all metrics over baseline, and their combination with CycleGAN achieved the best, yet modest results. \\
These findings indicate that appearance-level synthesis via CycleGAN offers little benefit for object detection and that gains observed in classification and segmentation does not necessarily translate into gains in object detection. \\
Future work should explore anatomically consistent, diffusion- or 3D style-based generators to better capture geometric variability for precise detection. \\
Limitations include reliance on a single dataset, one GAN architecture, and one object detection model, as well as 2D slice processing and absence of clinical validation of realism.
%
% ---- Bibliography ----
%
% BibTeX users should specify bibliography style 'splncs04'.
% References will then be sorted and formatted in the correct style.
%
\bibliographystyle{splncs04}
\bibliography{references}
\end{document}
