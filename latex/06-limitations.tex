\section{Limitations}
This work has several limitations. The evaluation is limited to a single dataset (LUNA16), a single GAN architecture (CycleGAN) and only one object detection model (YOLO11), which limits the generalizability of the findings. \\
Furthermore, the experiments are conducted on two-dimensional CT slices rather than full three-dimensional slices, which may limit the ability of both classical and GAN-based augmentation methods to capture inter-slice contextual information relevant for pulmonary nodule detection. \\
Additionaly, while CycleGAN-based augmentation increases appearance variability, it does not introduce geometric variations, which are critical for accurate object localization and may reduce their usefulness for improving such localization. \\
Furthermore, the realism of GAN-generated images is not assessed by clinical experts, although just appearance-level shifts, leaves the perceptual and diagnostic validity of the synthetic images unverified. \\
Beyond these limitations, computational contraints such as training CycleGAN on a capped number of images and only using subsets of the original dataset may have underutilized the potential of GAN-based augmentation. \\
Finally, while fixed hyperparameters and a single validation split, while ensuring comparability, may not reflect optimal hyperparameters or dataset variability.