\section{Related Work}
\subsection{GANs in Medical Imaging} % Tobi
GANs have become an effective tool in medical imaging for image synthesis, segmentation and classification. Frid-Adar et al.~\cite{frid2018gan} demonstrated that GAN-generated synthetic CT images of liver lesions improved classification. Sandfort et al.~\cite{sandfort2019data} utilized CycleGAN to adapt between contrast and non-contrast CT images, improving segmentation performance. Singh and Raza~\cite{singh2021medical} reviewed GAN architectures and emphasized their potential to overcome data scarcity. However, few studies have utilized GANs directly to object detection.
\subsection{Object Detection in Medical Imaging} % Tobi
Deep learning-based approaches have seen widespread use in medical object detection~\cite{albuquerque2025deep}\cite{yang2021artificial}. However, performance is limited due to the low availability of annotated medical images. To overcome this data scarcity, GAN-generated images can be generated to augment scarce datasets.
\subsection{Positioning of this Work} % Tobi
While GAN-based data augmentation has proven effective for image classification and segmentation, its impact on object detection is underexplored. This work adresses the gap by applying GAN-generated CT images to augment a medical images dataset and evaluate their effect on object detection performance.