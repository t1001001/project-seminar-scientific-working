\section{Limitations}
This work has several limitations. The evaluation is limited to a single dataset (LUNA16), a single GAN architecture (CycleGAN) and only one object detection model (YOLO11), which limits the generalizability of the findings. \\
Furthermore, the experiments are conducted on two-dimensional CT slices rather than full three-dimensional slices, which may limit the ability of both classical and GAN-based augmentation methods to capture inter-slice contextual information relevant for pulmonary nodule detection. Additionaly, while CycleGAN-based augmentation increases appearance variability, it does not explicitly model geometric variations, which are critical for accurate object localization and may reduce their usefulness for improving such localization. Furthermore, no statistical significance testing was conducted, and the observed performance differences may fall within the variance introduced by random initialization. \\
Lastly, the realism of GAN-generated images is not assessed by clinical experts.